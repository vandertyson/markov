\documentclass[14pt,a4paper,oneside]{report}		%lớp văn bản
\usepackage[utf8]{vietnam}						%gói ngôn ngữ tiếng Việt
%--
\usepackage{amsfonts}
\usepackage{latexsym, amsmath, amsxtra, amssymb, amscd, amsthm}	%gói ký tự toán
\usepackage{indentfirst}
\usepackage{fancyheadings}
\usepackage{color,colortbl}		%gói màu
\usepackage{graphicx}			%gói hình ảnh 
\usepackage{hyperref}			%gói liên kết link
\usepackage[top=3.5cm, bottom=3.0cm, left=3.5cm, right=2cm] {geometry}
\lhead{Mẫu báo cáo}
\rhead{nguoicontraiphonui.blogspot.com}
%//================================= Begin dinh nghia cac goi lenh
\renewcommand{\contentsname}{Mục lục}
\renewcommand{\chaptername}{Chương}
\renewcommand\bibname{Tài liệu tham khảo}
\newcommand{\gach}{\backslash}
\newtheorem{theorem}{Định lý}
%==================================// End dinh nghia cac goi lenh
%//================================== Begin make title
\title{{\bf  BÁO CÁO CUỐI KỲ MÔN MÔ HÌNH NGẪU NHIÊN VÀ ỨNG DỤNG}}
\author{Tác giả: nguoicontraiphonui \hspace*{1cm}Website: http://nguoicontraiphonui.blogspot.com \and \\
Soạn thảo văn bản \LaTeX{} bởi công cụ MikTeX $\&$ TeXmaker\\\\}
\date{{\em \today}}
%==================================// End make  title
\begin{document}
\pagestyle{fancy}	
\Large												%co chu
\maketitle											%make title
\setlength{\baselineskip}{5truept}		%gian dong cho muc luc
\tableofcontents									%tao muc luc
\newpage
\setlength{\baselineskip}{18truept}	%gian dong cho van ban
%//==========================================Begin noi dung bai==
\chapter*{Lời nói đầu}
\addcontentsline{toc}{chapter}{\quad\  \bf Lời nói đầu}
Đây là lời nói đầu
%==============================
\chapter{Chương mở đầu}
Chương đầu có nội dung như sau

\section{Phần đầu}
Viết vài lời về phần đầu, bao gồm những phần nhỏ như sau
\subsection{{\Large Phần nhỏ hơn đầu}}
viết vào đây....

\subsection{Phần nhỏ hơn tiếp}
viết đây....
\section{Phần tiếp}
Viết chút vào đây...

%============================
\chapter{Chương tiếp}
nội dung trang tiếp trình bày ở đây
\section{Phần đầu chương tiếp}
\newtheorem{vd}{Ví dụ}		%--dinh nghia dang theorem moi (trinh bay giong dang theorem)
\begin{vd}
Ví dụ minh họa....
\end{vd}

\begin{vd}
Ví dụ minh họa tiếp....
\end{vd}

\section{Phần tiếp}
Nội dung viết vào đây

%============================
\chapter*{Kết luận}
\addcontentsline{toc}{chapter}{\quad\ \bf Kết luận}
%============================
%Tài liệu tham khảo
\begin{thebibliography}{12}
\addcontentsline{toc}{chapter}{\quad\  \bf Tài liệu tham khảo}
\bibitem{1}Họ và tên tác giả, năm, {\it Tên sách} NXB.
\bibitem{2}Họ và tên tác giả, năm, {\it Tên sách} NXB.
\bibitem{3}Họ và tên tác giả, năm, {\it Tên sách} NXB.
\bibitem{4}Họ và tên tác giả, năm, {\it Tên sách} NXB.
\bibitem{5}Họ và tên tác giả, năm, {\it Tên sách} NXB.
\end{thebibliography}

%============================
\chapter*{Thanks for reading!}
Thanks for reading!
%===========================================End noi dung bai==//
\end{document}